\documentclass[type=bachelor, AutoFakeBold=2.5, fontset=windowsold]{thuthesis}
% 选项:
%   type=[bachelor|master|doctor|postdoctor], % 必选
%   secret,                                   % 可选
%   pifootnote,                               % 可选(建议打开)
%   openany|openright,                        % 可选,基本不用
%   arial,                                    % 可选,基本不用
%   arialtoc,                                 % 可选,基本不用
%   arialtitle                                % 可选,基本不用

% 所有其它可能用到的包都统一放到这里了,可以根据自己的实际添加或者删除。
\usepackage{thuthesis}

% 自定义包和命令
\usepackage{custom}

% 定义所有的图片文件在 figures 子目录下
\graphicspath{{figures/}}

% 可以在这里修改配置文件中的定义。导言区可以使用中文。
% \def\myname{薛瑞尼}

%% 多参考文献设定
\usepackage{bibunits}
\defaultbibliographystyle{thuthesis} % 默认样式
\defaultbibliography{ref/refs} % 默认 bib 文件路径

\makeatletter
\newcommand*{\newbibstartnumber}[1]{%
  \apptocmd{\thebibliography}{%
    \global\c@NAT@ctr #1\relax
    \addtocounter{NAT@ctr}{-1}%
  }{}{}%
}
\makeatother

\begin{document}

%%% 封面部分
\frontmatter
\input{data/cover}
% 如果使用授权说明扫描页,将可选参数中指定为扫描得到的 PDF 文件名,例如:
% \makecover[scan-auth.pdf]
\makecover

%% 目录
\tableofcontents

%% 符号对照表
\input{data/denotation}


%%% 正文部分
\mainmatter

\begin{bibunit}

% 正文文件
\include{data/chap01}
\include{data/chap02}


%%% 其它部分
\backmatter

%% 本科生要这几个索引,研究生不要。选择性留下。
% 插图索引
\listoffigures
% 表格索引
\listoftables
% 公式索引
\listofequations

\putbib

\end{bibunit}

%% 致谢
\include{data/ack}

%% 附录
\newbibstartnumber{1} %重置参考文献计数器

\begin{bibunit}

\begin{appendix}

% 附录文件
\input{data/appendix01}

\renewcommand{\bibname}{附录参考文献}
\putbib

\end{appendix}

\end{bibunit}


%% 个人简历
\include{data/resume}

%% 本科生进行格式审查是需要下面这个表格,答辩可能不需要。选择性留下。
% 综合论文训练记录表
\includepdf[pages=-]{scan-record.pdf}
\end{document}
